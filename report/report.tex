\documentclass[]{elsarticle} %review=doublespace preprint=single 5p=2 column
%%% Begin My package additions %%%%%%%%%%%%%%%%%%%
\usepackage[hyphens]{url}
\usepackage{lineno} % add
\providecommand{\tightlist}{%
  \setlength{\itemsep}{0pt}\setlength{\parskip}{0pt}}

\bibliographystyle{elsarticle-harv}
\biboptions{sort&compress} % For natbib
\usepackage{graphicx}
\usepackage{booktabs} % book-quality tables
%% Redefines the elsarticle footer
%\makeatletter
%\def\ps@pprintTitle{%
% \let\@oddhead\@empty
% \let\@evenhead\@empty
% \def\@oddfoot{\it \hfill\today}%
% \let\@evenfoot\@oddfoot}
%\makeatother

% A modified page layout
\textwidth 6.75in
\oddsidemargin -0.15in
\evensidemargin -0.15in
\textheight 9in
\topmargin -0.5in
%%%%%%%%%%%%%%%% end my additions to header

\usepackage[T1]{fontenc}
\usepackage{lmodern}
\usepackage{amssymb,amsmath}
\usepackage{ifxetex,ifluatex}
\usepackage{fixltx2e} % provides \textsubscript
% use upquote if available, for straight quotes in verbatim environments
\IfFileExists{upquote.sty}{\usepackage{upquote}}{}
\ifnum 0\ifxetex 1\fi\ifluatex 1\fi=0 % if pdftex
  \usepackage[utf8]{inputenc}
\else % if luatex or xelatex
  \usepackage{fontspec}
  \ifxetex
    \usepackage{xltxtra,xunicode}
  \fi
  \defaultfontfeatures{Mapping=tex-text,Scale=MatchLowercase}
  \newcommand{\euro}{€}
\fi
% use microtype if available
\IfFileExists{microtype.sty}{\usepackage{microtype}}{}
\ifxetex
  \usepackage[setpagesize=false, % page size defined by xetex
              unicode=false, % unicode breaks when used with xetex
              xetex]{hyperref}
\else
  \usepackage[unicode=true]{hyperref}
\fi
\hypersetup{breaklinks=true,
            bookmarks=true,
            pdfauthor={},
            pdftitle={Constructing Bipartite Backbones in Wireless Sensory Networks},
            colorlinks=true,
            urlcolor=blue,
            linkcolor=magenta,
            pdfborder={0 0 0}}
\urlstyle{same}  % don't use monospace font for urls
\setlength{\parindent}{0pt}
\setlength{\parskip}{6pt plus 2pt minus 1pt}
\setlength{\emergencystretch}{3em}  % prevent overfull lines
\setcounter{secnumdepth}{0}
% Pandoc toggle for numbering sections (defaults to be off)
\setcounter{secnumdepth}{0}
% Pandoc header


\usepackage[nomarkers]{endfloat}

\begin{document}
\begin{frontmatter}

  \title{Constructing Bipartite Backbones in Wireless Sensory Networks}
    \author[Southern Methodist University - Bobby B. Lyle School of Engineering]{Will Spurgin}
   \ead{wspurgin@smu.edu} 
  
    
  \begin{abstract}
  TODO FILL IN ABSTRACT
  \end{abstract}
  
 \end{frontmatter}

\section{Executive Summary}\label{executive-summary}

\subsection{Introduction}\label{introduction}

Imagine a scenario in which you are part of a team monitoring the
volcano Mount Pinatubo for surface temperature and seismic activity.
Further, you are given a multitude of low cost short-range wireless
sensors to place around an area of the volcano to take these
measurements. Since Mt. Pinatubo is somewhat hard to reach, you decide
to drop these sensors from a helicopter. After dropping your sensors
over the area in a uniformly random fashion, you now need to read all
the data from these sensors. However, since they are short-range
wireless sensors, you would have to be sufficiently close to a sensor to
read its data. However, since all the nodes are wireless, they can
communicate their data to each other. Ideally, you would rather go to
one or relatively few nodes to retrieve all the data. The question
becomes: to which node(s) do you go?

This fictional scenario is the problem setting to which this report
provides solution discussion, reduction of those solution to practice,
and a variety of benchmark measurements. Formally, this report examines
the construction of backbones of communication in Wireless Sensor
Networks (WSNs) modeled as Random Geometric Graphs (RGGs). The results
presented in this report indicate that, generally, RGGs (and thus WSNs)
with uniformly distributed nodes can achieve near \textbf{100\%}
coverage from a bipartite backbone using the methods described in this
report. The use of the methods presented in this report are strongly
recommended for any application involving WSNs or scenarios that can be
modeled as RGGs. Study of this work is not new{[}1,2{]}, and much of
that work corroborates the findings of this report.

In the presentation of this report, several tables, and graphics are
used to make the seeing the merit of the results effortless. For all
benchmarks, the following plots are included: A plot of the RGG (with
edges), a plot of the node with lowest degree, a plot of the node with
highest degree, a plot of the color classes frequency, a plot of the
degree distributions, a plot of the degree when deleted versus original
degree of the Smallest Last ordering algorithm, and lastly the plot of
the two major backbones of an RGG. Additionally, a table is provided
with summary information about the RGG (e.g.~number of edges, vertices,
desired average degree, actual average degree, etc.). An abbreviated
table of results is given below in Table \(\ref{tab:abb_res}\).

\begin{table}

\caption{\label{tab:abb_res}Abbreviated Results}
\centering
\begin{tabular}[t]{l|l}
\hline
Number of Nodes & 29464\\
\hline
Desired Avg. Degree & 32\\
\hline
Shape & plane\\
\hline
Actual Avg. Degree & 29.464\\
\hline
Radius & 0.100925\\
\hline
Number of Edges & 29464\\
\hline
Max Degree & 48\\
\hline
Min Degree & 7\\
\hline
\end{tabular}
\end{table}

\subsection{Environment Description}\label{environment-description}

\begin{itemize}
\tightlist
\item
  Hardware:

  \begin{itemize}
  \tightlist
  \item
    Apple MacBook Pro (Retina, Mid 2012)
  \item
    Processor: 2.6 GHz Intel Core i7 (with turbo boost up to 3.3GHz)
  \item
    Memory: 8 GB 1600 MHz DDR3
  \item
    Intel HD Graphics 4000 1536 MB
  \item
    Storage: 500.28 GB Solid State Drive, SATA Connection.
  \end{itemize}
\item
  Language: C++

  \begin{itemize}
  \tightlist
  \item
    Compilers:
  \item
    GNU GCC 5.2.0 (Homebrew gcc 5.2.0 build)
  \item
    Clang 8, Apple LLVM version 8.0.0 (clang-800.0.38)
  \end{itemize}
\item
  Graphical Tool:

  \begin{itemize}
  \tightlist
  \item
    R{[}3{]} version 3.3.1 (2016-06-21) -- (codename ``Bug in Your
    Hair'')
  \end{itemize}
\item
  Resource Utilization:

  \begin{itemize}
  \tightlist
  \item
    CPU exponential from 0\% to 16\%
  \item
    Memory 0 MB to 40 MB
  \end{itemize}
\end{itemize}

\section{Evaluation}\label{evaluation}

Nullam semper imperdiet orci, at lacinia est aliquet et. Sed justo nibh,
aliquet et velit at, pharetra consequat velit. Nullam nec ligula
sagittis, adipiscing nisl sed, varius massa. Mauris quam ante, aliquet a
nunc et, faucibus imperdiet libero. Suspendisse odio tortor, bibendum
vel semper sit amet, euismod ac ante. Nunc nec dignissim turpis, ac
blandit massa. Donec auctor massa ac vestibulum aliquam. Fusce auctor
dictum lobortis. Vivamus tortor augue, convallis quis augue sit amet,
laoreet tristique quam. Donec id volutpat orci. Suspendisse at mi vel
elit accumsan porta ac ut diam. Nulla ut dapibus quam.

Sed est odio, ornare in rutrum et, dapibus in urna. Suspendisse varius
massa in ipsum placerat, quis tristique magna consequat. Suspendisse non
convallis augue. Quisque fermentum justo et lorem volutpat euismod.
Vestibulum ante ipsum primis in faucibus orci luctus et ultrices posuere
cubilia Curae; Morbi sagittis interdum justo, eu consequat nisi
convallis in. Sed tincidunt risus id lacinia ultrices. Phasellus ac
ligula sed mi mattis lacinia ac non felis. Etiam at dui tellus.

\section{Conclusion}\label{conclusion}

Duis nec purus sed neque porttitor tincidunt vitae quis augue. Donec
porttitor aliquam ante, nec convallis nisl ornare eu. Morbi ut purus et
justo commodo dignissim et nec nisl. Donec imperdiet tellus dolor, vel
dignissim risus venenatis eu. Aliquam tempor imperdiet massa, nec
fermentum tellus sollicitudin vulputate. Integer posuere porttitor
pharetra. Praesent vehicula elementum diam a suscipit. Morbi viverra
velit eget placerat pellentesque. Nunc congue augue non nisi ultrices
tempor.

\section*{References}\label{references}
\addcontentsline{toc}{section}{References}

\hypertarget{refs}{}
\hypertarget{ref-mahjoub2010}{}
{[}1{]} Mahjoub D, Matula DW. Building (1-\(\epsilon\)) dominating sets
partition as backbones in wireless sensor networks using distributed
graph coloring. In: Rajaraman R, Moscibroda T, Dunkels A, Scaglione A,
editors. Distributed computing in sensor systems: 6th ieee international
conference, dcoss 2010, santa barbara, ca, usa, june 21-23, 2010.
proceedings, Berlin, Heidelberg: Springer Berlin Heidelberg; 2010, pp.
144--57.
doi:\href{https://doi.org/10.1007/978-3-642-13651-1_11}{10.1007/978-3-642-13651-1\_11}.

\hypertarget{ref-mahjoub2012}{}
{[}2{]} Mahjoub D, Matula DW. Constructing efficient rotating backbones
in wireless sensor networks using graph coloring. Computer
Communications 2012;35:1086--97.
doi:\href{https://doi.org/http://dx.doi.org/10.1016/j.comcom.2012.02.013}{http://dx.doi.org/10.1016/j.comcom.2012.02.013}.

\hypertarget{ref-r2015}{}
{[}3{]} Team RC. R: A language and environment for statistical
computing. Vienna, Austria: R Foundation for Statistical Computing;
2015.

\end{document}


